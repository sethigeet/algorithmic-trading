\chapter{Fundamentals of Stock Markets}

\section{Types of Assets (Asset Classes)}

\dfn{Asset Class}{An asset class is a category of investment with particular risk and return characteristics. The following are some of the popular asset classes:
\begin{itemize}
  \item Fixed income instruments
  \item Equity
  \item Real estate
  \item Commodities (e.g. precious metals)
\end{itemize}}

\subsection{Fixed Income Instruments}
These carry very limited risk to the principle and the return is paid as an interest based on the particular fixed income instrument.
\\
Interest paid could be at quarterly, semi-annual or annual intervals.
The capital is returned to the investor at the end of the term of the deposit.
\\
Typical fixed income investments include:
\begin{itemize}
  \item Fixed deposits
  \item Bonds issued by the government and its agencies
  \item Bonds issued by corporates
\end{itemize}

\subsection{Equity}
Investment in equities involve buying shares of publicly listed companies.
\\
The shares are traded both on the Bombay Stock Exchange (BSE) and the National Stock Exchange (NSE).
\\
Unlike fixed income securities, these offer no guarantee against the capital. However, as a trade off, these investments can yield very attractive returns. Indian equities have generated returns close to 14\%-15\% CAGR (compound annual growth rate).

\newpage

\subsection{Real Estate}
Real estate investment involves transacting commercial and non-commercial land (e.g. sites, apartments, commercial buildings, etc.).
\nt{There is no official metric to measure the returns generated by real estate.}

\subsection{Commodity}
Investments in precious metals such as gold and silver is considered one of the most popular investment avenues.
\\
Gold and silver over a long-term period have yielded a CAGR return of approximately 8\% over the last 20 years.

\section{Financial Intermediaries}
\subsection{The Regulator}
In India, the stock market is regulated by the \textbf{The Securities and Exchange Board of India} also referred to as \textbf{SEBI}. The main objective of SEBI is to promote the development of stock exchanges, protect the interests of retail investors, regulate the activates of market participants and financial intermediaries.

\subsection{The Stock Broker}
\dfn{Stock Broker}{A stock broker is a private entity, registers as a trading member with the stock exchange and a stock broking license. They are a gateway to the stock exchanges. An individual must go through a stock broker to buy/sell stocks.}
An individual who wishes to trade in the stock market must open a "Trading Account" with a broker through which they can then trade at the stock exchanges.

\subsection{Depository and Depository Participants}
Earlier, when you bought stocks, the only way to identify that you owned the stock was a piece of paper called the share certificate. Hence it became extremely important to store the property papers in a safe and secure place.
\\
Seeing the obvious problem of storing the certificates, after 1996, the share certificates were converted to a digital format.
\dfn{Dematerialization}{The process of converting paper format share certificates to digital format is called dematerialization and is often abbreviated as \textbf{DEMAT}}

The share certificate, though in digital form, still needs to be stored securely. This is done through DEMAT accounts.

\dfn{Depository}{A depository is a financial intermediary which offers the service of a DEMAT account.}

DEMAT accounts act as a digital vault for your shares.

\subsection{Banks}
Banks help in facilitating the fund transfer from your bank account to your trading account.

\subsection{Clearing Corporations}
National Security Clearing Corporation Ltd (NSCCL) and Indian Clearing Corporation Ltd (ICCL) are wholly owned subsidiaries of NSE and BSE respectively.
\\
The job of the clearing corporation is to ensure guaranteed settlement of your trades/transactions.
\\
The typical roles of the clearing corporation is to ensure the following:
\begin{itemize}
  \item Identify the buyer and seller and match the debit and credit process
  \item Ensure no defaults - The clearing corporation also ensures there are no defaults by either party.
\end{itemize}

\section{Calculating Returns}

\subsection{Absolute Return}
This is the return that your investment has generated in absolute terms.
\[ \left( \frac{\textrm{Ending Period Value}}{\textrm{Starting Period Value}} - 1 \right) \times 100 \]
\ex{Calculating absolute return}{
  \textbf{Let's say that you bought a stock at 3030 and sold it at 3550. What absolute return did you generate?}
  \begin{displaymath}
    \begin{split}
      \textrm{absolute return} &= \left( \frac{3550}{3030} - 1 \right) \times 100 \\
                               &= 0.1716 \times 100 \\
                               &= 17.16\%
    \end{split}
  \end{displaymath}
}

\subsection{Compound Annual Growth Rate (CAGR)}
The formula to calculate CAGR is:
\begin{displaymath}
  \textrm{CAGR} = \left( \frac{\textrm{Ending Value}}{\textrm{Beginning Value}} \right)^{\left( \frac{1}{\textrm{No. of years}} \right)} - 1
\end{displaymath}

\ex{Calculating CAGR}{
  \textbf{Let's say that you bought a stock at 3030 and sold it at 3550. What CAGR did you generate?}
  \begin{displaymath}
    \begin{split}
      \textrm{CAGR} &= \left(\frac{3550}{3030} \right) ^ \frac{1}{2} - 1 \\
                    &= 9.2 - 1 \\
                    &= 8.2\%
    \end{split}
  \end{displaymath}
  
}

\section{Index}
There are two main market indices in India. The \textbf{S\&P Sensex} representing the BSE and \textbf{CNX Nifty} representing the NSE.

\subsection{Practical Uses of the Index}
\begin{itemize}
  \item \textbf{Information}: The index reflects the general market trend for a period of time.
  \item \textbf{Benchmarking}: The index can be used to judge whether the returns you got over a period of time by comparing the returns to the increase in the index.
  \item \textbf{Trading}: Majority of the traders in the market trade the index.
  \item \textbf{Portfolio Hedging}: Investors usually build their own portfolio which typically contains 10-12 stocks which they would have bought from a long term perspective. If they can foresee a prolonged adverse movement in the market (such as in 2008) which could potentially erode the capital in the portfolio, the investors can use the index to hedge the portfolio.
\end{itemize}

\subsection{Index Construction Methodology}
Every stock in the index is assigned a certain weightage. There are many ways to calculate these weights but the Indian stock exchange follows a method called \textbf{free float market capitalization}. In the method, the larger the market capitalization of the company, higher its weight.
\[ \textrm{Free float market capitalization} = \textrm{total number of shares outstanding in the market} \times \textrm{price of the stock} \]

\subsection{Sector Specific Indices}
While the Sensex and Nifty represent the broader markets, there are certain indices that represent specific sectors. These are called \textbf{sectoral indices}. For example, Bank Nifty on NSE represents mood specific to the banking industry.

\section{Clearing \& Settlement Process}

\subsection{What happens when you buy a stock?}
\textit{\large{Day 1 - The trade (T Day)}}
\\
The day one makes a trade is referred to as the trade date (represented as 'T Day').
\\
By the end of the day, your broker will debit the amount required for the trade and other applicable charges towards the purchase.
\\
An important point to note is that the money is debited from your account but the stock does not come into your DEMAT account yet.
\\
The same day, the broker generates a 'contract note'. A contract note typically shows the break up of all transactions done during the day along with the trade reference number as well as the charges charged by the broker. \\
\\
\textit{\large{Day 2 - Trade Day + 1 (T+ Day)}}
\\
One can sell the stock that they bought on the trade day on this day.
\imp{This does involve a slight risk since you do not own the stock that you bought the previous day yet i.e. that stock hasn't been deposited into your DEMAT account.}{}
\noindent From the point of view of the user, nothing happens on this day. However, in the background the money required to purchase the shares is collected by the exchange along with other charges. \\
\\
\textit{\large{Day 3 - Trade Day + 2 (T+2 Day)}}
\\
On this day, around 11 AM, the shares are debited from the person who sold the shares and credited to the brokerage with whom the person is trading, who will in turn credit it to your DEMAT account by end of day. Similarly, money which was debited from your account is credited to the person who sold the shares.

\subsection{What happens when you sell a stock?}
Similar to the process which takes play when you buy a stock, the day when you sold the stock is called the 'trade day'. The moment you sell the stock, the stock gets blocked in your DEMAT account. On the T+1 day, the blocked shares are given to the exchange. On T+2 day, you would receive the funds from the sale which will be credited to your trading account.

\section{Orders in the Market}
\subsection{Types of Orders}

\begin{itemize}
  \item \textbf{Market Order}: A market order is \emph{buying or selling a stock at the best price available}. Generally, this type of order will be \emph{executed immediately}. However, the price at which the market order will be executed is not guaranteed. The \textit{last traded price} (LTP) need not be the price at which the order is executed.
  \item \textbf{Limit Order}: A limit order is an \emph{order to buy or sell a stock at a specific price or better}.  A limit order is not guaranteed to be executed. But they do help ensure the investor does not pay more than a predetermined price for a stock.
  \item \textbf{Stop-Loss Order}: A stop order, also referred to as stop-loss order, is an \emph{order to buy or sell a stock once the price reaches a specified price}, also known as the stop price. When the stop price is reached, a stop order becomes a market order. A buy-stop order is entered at a stop price which is above the current market price. Investors generally use a buy-stop order to limit a loss or to protect a profit on a stock that they have sold short. A sell-stop order is entered at a stop price below the current market price. Investors generally use a sell-stop order to limit a loss or to protect a profit on a stock that they own.
  \item \textbf{Stop-Limit Order}: A stop-limit order is an \emph{order to buy or sell a stock that combines the features of a stop order and a limit order}. Once the stop price is reached, a stop-limit order becomes a limit order that will be executed at specified price or better. The benefit of a stop-limit order is that the investor can control the price at which the order can be executed.
  \item \textbf{Take Profit Order}: A take-profit order (sometimes called a profit target) is \emph{intended to close out the trade at a profit once it has reached a certain level}. Execution of a take-profit order closes the position. This type of order is always connected to an open position of a pending order.
\end{itemize}

\subsection{Slippage}
\dfn{Slippage}{Slippage refers to the difference between the expected price of a trade and the price at which the trade is executed.}
Slippage can occur at any time but is most prevalent during periods of higher volatility when market orders are used. It can also occur when a large order is executed but there isn't enough volume at the chosen price to maintain the current bid/ask spread. \\
An $x$\% of slippage means the order was executed $x$\% below or above the expected price. \\
Disadvantages of high slippage include:
\begin{itemize}
  \item Increased trading costs
  \item Reduced profitability
  \item Inaccurate risk management
  \item Difficulty in entering and exiting positions
\end{itemize}

\subsection{Risk-Reward Ratio}
This ratio is used to assess the potential profitability and risk of a trade or investment opportunity. It is a way to evaluate the relationship between the potential reward of a trade and the amount of risk taken. \\
\\
The risk-reward ratio is calculated by dividing the potential reward (or profit) of a trade by the potential risk (or loss). The resulting ratio provides an indication of how much profit is expected for each unit of risk assumed.

