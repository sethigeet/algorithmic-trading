\chapter{Fundamental Analysis}

\section{Overview}

Fundamental Analysis (FA) is a holistic approach to study a business. When an investor wishes to invest in a business for the long term(say 3 - 5 years) it becomes extremely essential to understand the business from various perspectives. It is essential for the investor to separate the daily short term noise in the stock prices and concentrate on the underlying business performance. Over the long term, the stock prices of a fundamentally string companies tend to appreciate.

\section{Investible Grade Attributes}
An investible grade company has a few distinguishable characteristics. These characteristics can be classified user two heads namely the 'Qualitative aspect' and the 'Quantitative aspects'.

\subsection{Qualitative Aspects}
These mainly involve understanding the non-numeric aspects of the business. It includes many factors such as:
\begin{itemize}
  \item Management's background
  \item Business ethics - is the managements involved in scams, bribery, unfair business practices, etc.
  \item Corporate governance - appointments of directors, organization structure, transparency, etc.
  \item Minority shareholders - how does the management treat its minority shareholders, do they consider their interests while taking corporate actions
  \item Related party transactions - Is the company tendering financial favors to known entities such as promoter's relatives, friends, vendors, etc. at the cost of the shareholders funds?
  \item Salaries paid to promoters - Is the management paying themselves a hefty salary, usually a percentage of profits
  \item Operator activity in stocks - Does the stock price display unusual price behavior especially at the time when the promoter is transacting in the shares.
  \item Shareholders - Who are the significant shareholders in the firm
  \item Political affiliation
  \item Promoter lifestyle - Are the promoters too flamboyant and loud about their lifestyle?
\end{itemize}

\subsection{Quantitative Aspects}
These are matters related to financial numbers. These include many things, to name a few:
\begin{itemize}
  \item Profitability and its growth
  \item Margins and its growth
  \item Earnings and its growth
  \item Matters related to expenses
  \item Operating efficiency
  \item Pricing power
  \item Matters related to taxes
  \item Dividends payout
  \item Cash flow from various activities
  \item Debt – both short term and long term
  \item Working capital management
  \item Asset growth
  \item Investments
  \item Financial Ratios
\end{itemize}

\nt{The list is actually endless. In fact, each sector has many different metrics which are relevant to that sector only.}{For example, in the retail industry, the following metrics are used:
\begin{itemize}
  \item Total number of stores
  \item Average sales per store
  \item Total sales per square foot
  \item Merchandise margins
  \item Owned store to franchisee ratio
\end{itemize}}

\section{Annual Report}
The annual report (AR) is a yearly publication by the company and is sent to the shareholders and other interested parties. The annual report is published by the end of the Financial Year, and all the data made available in the annual report is dated to 31st March.

\imp{No two annual reports are the same; they are all made to suite the company's requirement keeping in perspective the industry they operate in. However, some sections are common across all annual reports.}

The annual report also contains three important financial statements namely:
\begin{itemize}
  \item Profit and Loss Statement
  \item Balance Sheet
  \item Cash Flow Statement
\end{itemize}

\section{The Profit and Loss Statement}

\subsection{Overview}
The Profit and Loss statement is also popularly referred to as the P\&L statement, Income Statement, Statement of Operations, and Statement of Earnings. \\
The P\&L statement reports information on:
\begin{itemize}
  \item The revenue of the company for the given period (yearly or quarterly)
  \item The expenses incurred to generate the revenues
  \item The tax and depreciation
  \item The earnings per share number
\end{itemize}

\subsection{Some Jargon Used in these Statements}
\begin{itemize}
  \item \textbf{Top Line}: The top line of the company is the revenue generated by the company.
  \item \textbf{Net Sales}: The revenue adjusted after the excise duty is the net sales of the company.
  \item \textbf{Total Operating Revenue}: Revenue from sales of products + sale if services + other operating revenues sums up to give the total operating revenue of the company.
  \item \textbf{Bottom Line}: 
  \item \textbf{Tangible asset}: This asset is one which has a physical form and provides an economic value to the company.
  \item \textbf{Intangible asset}: This asset is one which does not have any physical form but still provides an economic value to the company such as brand value, trademarks, etc.
  \item \textbf{Deprecation and Amortization}: An asset (tangible or intangible) has to be depreciated over its useful life. Useful life is defined as the period during which the asset can provide economic benefit to the company. Since an asset would continue to provide economic benefits over its useful life, it makes sense to spread the cost of acquiring the asset over its useful life. This is called depreciation. The deprecation equivalent for intangible assets is called amortization.
  \item \textbf{Profit After Tax (PAT)}
  \item \textbf{Earnings Per Share (EPS)}: EPS reflects the earning capacity of a company on a per share basis. $ \textrm{EPS} = \frac{\textrm{PAT}}{\textrm{Total number of outstanding ordinary shares}} $
\end{itemize}

\section{The Balance Sheet}

\dfn{Asset}{An asset is a resource controlled by the 
company, and is expected to have an economic value in the future. Typical examples of assets include plants, machinery, cash, brands, patents etc. Assets are of two types, current and non-current.}

\dfn{Liability}{A liability represents the company's obligation.  The obligation is taken up by the company because the company believes these obligations will provide economic value in the long run. Liability in simple words is the loan that the company has taken and it is therefore obligated to repay back.}

\nt{In a typical balance sheet, the total assets of a company should be equal to the total liabilities of the company.}

\dfn{Owners Capital}{It is the difference between the assets and the liabilities. \\
It is also called the 'Shareholders Equity' or the 'Net Worth'}

\subsection{Types of Assets \& Liabilities}
\subsubsection{Current Assets}
Current assets are assets that can be easily converted to cash and the company foresees a situation of consuming these assets within 365 days. Current assets are the assets that a company uses to fund its day to day operations and ongoing expenses.
\subsubsection{Non-Current Assets (Fixed Assets)}
Non-current assets are the assets that the company owns, the economic benefit of which is enjoyed over a long period (beyond 365 days)
\subsubsection{Current Liabilities}
Current liabilities are a company’s obligations which are expected to be settled within 365 days (less than 1 year). The term ‘Current’ is used to indicate that the obligation is going to be settled soon, within a year.
\subsubsection{Non-Current Liabilities}
Non-current liabilities represent the long term obligations, which the company intends to settle/pay off not within 365 days/ 12 months of the balance sheet date. These obligations stay on the books for few years. Non-current liabilities are generally settled after 12 months after the reporting period.

\section{The Cash Flow Statement}
\subsubsection{Overview}
The cash flow statement provides information to the users of the financial statements about the entity’s ability to generate cash and cash equivalents as well as indicates the cash needs of a company.

\subsection{Activities Undertaken by a Company}
Any legitimate company has three main activities:
\begin{itemize}
  \item \textbf{Operational activities (OA)}: Activities that are directly related to the daily core business operations are called operational activities. Typical operating activities include sales, marketing, manufacturing, technology upgrade, resource hiring etc.
  \item \textbf{Investing activities (IA)}: Activities pertaining to investments that the company makes with an intention of reaping benefits at a later stage. Examples include parking money in interest bearing instruments, investing in equity shares, investing in land, property, plant and equipment, intangibles and other non current assets etc.
  \item \textbf{Financing activities (FA)}: Activities pertaining to all financial transactions of the company such as distributing dividends, paying interest to service debt, raising fresh debt, issuing corporate bonds etc.
\end{itemize}

\section{Financial Ratios}

\subsection{Profitability Ratios}
The Profitability ratios help the analyst measure the profitability of the company. The ratios convey how well the company is able to perform in terms of generating profits.

\subsubsection{EBITDA Margin}
The Earnings before Interest Tax Depreciation \& Amortization (EBITDA) Margin tells us how profitable (in percentage terms)the company is at an operating level.

\begin{align*}
  & \textrm{EBITDA} = \textrm{Operating Revenues} - \textrm{Operating Expenses} \\
  & \textrm{Operating Revenues} = \textrm{Total Revenue} - \textrm{Other Income} \\
  & \textrm{Operating Expenses} = \textrm{Total Expense} - \textrm{Finance Cost} - \textrm{Depreciation \& Amortization} \\
  & \textrm{EBITDA Margin} = \frac{\textrm{EBITDA}}{\textrm{Total Revenue} - \textrm{Other Income}} \\
\end{align*}

\subsubsection{PAT Margin}
While the EBITDA margin is calculated at the operating level, the Profit After Tax (PAT) margin is calculated at the final profitability level. At the operating level we consider only the operating expenses however there are other expenses such as depreciation and finance costs which are not considered. Along with these expenses there are tax expenses as well. When we calculate the PAT margin, all expenses are deducted from the Total Revenues of the company to identify the overall profitability of the company.
\begin{displaymath}
  \textrm{PAT Margin} = \frac{\textrm{PAT}}{\textrm{Total Revenue}}
\end{displaymath}

\subsubsection{Return on Equity (RoE)}
It is the return the shareholder earns for every unit of capital invested. RoE measures the entity’s ability to generate profits from the shareholders investments.
\begin{displaymath}
  \textrm{RoE} = \frac{\textrm{Net Profit}}{\textrm{Shareholders Equity}} \times 100
\end{displaymath}

\imp{DuPont Model}{Inspecting the RoE closely is very important because as the company takes on more debt instead of investment through equity, the RoE shoots up but that debt is not good for the company! \\
To combat this, \emph{DuPont} came up with another way of writing the RoE: 
\begin{displaymath}
  \textrm{RoE} = \frac{\textrm{Net Profit}}{\textrm{Net Sales}} \times \frac{\textrm{Net Sales}}{\textrm{Avg. Total Assets}} \times \frac{\textrm{Avg. Total Assets}}{\textrm{Shareholder Equity}}
\end{displaymath}

If you notice, the terms cancel out to give the original formula back. However, in this process of decomposing the formula, one gained insights into three distinct aspects of the company's business.
}

\subsection{Leverage Ratios}
The Leverage ratios also referred to as solvency ratios/ gearing ratios measures the company’s ability (in the long term) to sustain its day to day operations. Leverage ratios measure the extent to which the company uses the debt to finance growth.

\subsubsection{Interest Coverage Ration}
The interest coverage ratio, also referred to as the debt service ratio or the debt service coverage ratio, helps us understand how much the company is earning relative to the interest burden on the company. Hence, it helps us interpret how easily can the company pay its interest payments.
\begin{displaymath}
  \textrm{Interest Coverage Ration} = \frac{\textrm{Earnings before interest and tax}}{\textrm{Interest Payment}}
\end{displaymath}

\subsubsection{Debt to Equity Ratio}
It measures the amount of total debt capital with respect to the total equity capital.

\begin{displaymath}
  \textrm{Debt to Equity Ratio} = \frac{\textrm{Total Debt}}{\textrm{Total Equity}}
\end{displaymath}

\subsubsection{Debt to Asset Ratio}
It measures the amount of total debt capital with respect to the total equity capital.

\begin{displaymath}
  \textrm{Debt to Asset Ratio} = \frac{\textrm{Total Debt}}{\textrm{Total Assets}}
\end{displaymath}

\subsubsection{Financial Leverage Ratio}
The financial leverage ratio gives us an indication to what extent the assets are supported by equity.

\begin{displaymath}
  \textrm{Financial Leverage Ratio} = \frac{\textrm{Average Total Asset}}{\textrm{Average Total Equity}}
\end{displaymath}

\subsection{Valuation Ratios}
The Valuation ratios compare the stock price of the company with either the profitability of the company or the overall value of company to get a sense of how cheap or expensive the stock is trading.

\subsubsection{Price to Sales Ratio}
In many cases, investors may use sales instead of earnings to value their investments. The earnings figure may not be true as some companies might be experiencing a cyclical low in their earning cycle.

\begin{displaymath}
  \textrm{P/S Ratio} = \frac{\textrm{Current Share Price}}{\textrm{Sales per share}}
\end{displaymath}

\subsubsection{Price to Book Value Ratio}
\dfn{Book Value}{It is simply the amount of money left on the table after the company pays off all of its obligations. \\
This is the amount of money the company can expect to receive after it sells all of its assets and settles its debts. \\
\begin{displaymath}
  \textrm{BV} = \frac{\textrm{Share Capital} + \textrm{Reserves}}{\textrm{Total Number of Shares}}
\end{displaymath}
}
Many investors may choose to value a company based on how much its book value is because that is a guaranteed amount that the company will get for sure even in the worst case scenarios.

\begin{displaymath}
  \textrm{P/BV Ratio} = \frac{\textrm{Current Share Price}}{\textrm{Book Value per share}}
\end{displaymath}

\subsubsection{Price to Earning Ratio}
We know that the \emph{EPS} measures the profitability of a company on a per share basis. Dividing the current market price by the \emph{EPS} gives us the \emph{Price to Earnings ratio} of the firm. It measures the willingness of the market participants to pay for the stock, for every rupee of profit that the company generates.

\begin{displaymath}
  \textrm{P/E Ratio} = \frac{\textrm{Current Share Price}}{\textrm{Earnings per share}}
\end{displaymath}

\subsection{Operating Ratios}
The Operating Ratios, also called the ‘Activity Ratios’ measures the efficiency at which a business can convert its assets (both current and non-current) into revenues. This ratio helps us understand how efficient the management of the company is.

\subsubsection{Fixed Assets Turnover}
The ratio measures the extent of the revenue generated in comparison to its investment in fixed assets. It tells us how effectively the company uses its assets. Fixed assets include the property, plant and equipment. Higher the ratio, it means the company is effectively and efficiently managing its fixed assets.

\begin{displaymath}
  \textrm{Fixed Assets Turnover} = \frac{\textrm{Operating Revenues}}{\textrm{Total Average Asset}}
\end{displaymath}

\subsubsection{Working Capital Turnover}
\dfn{Working Capital}{Working capital refers to the capital required by the firm to run its day to day operations.}

\begin{displaymath}
  \textrm{Working Capital} = \textrm{Current Assets} - \textrm{Current Liabilities}
\end{displaymath}

The working capital turnover, also referred to as net sales to working capital, indicates how much revenue the company generates for every unit of working capital. Higher the number, the better it is.

\begin{displaymath}
  \textrm{Working Capital Turnover} = \frac{\textrm{Revenue}}{\textrm{Average Working Capital}}
\end{displaymath}

\subsubsection{Total Assets Turnover}
The ratio measures the extent of the revenue generated in comparison to its investment in fixed assets. It tells us how effectively the company uses its assets. Fixed assets include the property, plant and equipment. Higher the ratio, it means the company is effectively and efficiently managing its fixed assets.

\begin{displaymath}
  \textrm{Fixed Assets Turnover} = \frac{\textrm{Operating Revenues}}{\textrm{Total Average Asset}}
\end{displaymath}

\subsubsection{Receivables Turnover Ratio}
The receivable turnover ratio indicates how 
many times in a given period the company receives money/cash from its debtors and customers. Naturally a high number indicates that the company collects cash more frequently.

\begin{displaymath}
  \textrm{Receivables Turnover Ratio} = \frac{\textrm{Revenue}}{\textrm{Average Receivables}}
\end{displaymath}

\subsubsection{Days Sales Outstanding (DSO) / Average Collection Period}
The days sales outstanding ratio illustrates the average cash collection period i.e the time lag between billing and collection.

\begin{displaymath}
  \textrm{DSO} = \frac{365}{\textrm{Receivables Turnover Ratio}}
\end{displaymath}

\section{Valuation of a Company using Discounted Cash Flow (DCF) Method}
Valuation per say helps the individual determine the 'intrinsic value' of the company. We will now look at a valuation technique called the \textbf{Discounted Cash Flow (DCF)} analysis to calculate the intrinsic value of the company.

\subsection{Time Value of Money}
If we have to evaluate, what would be the value of money that we have today sometime in the future, then we need to move the ‘money today’ through the future. This is called the \textbf{Future Value (FV)} of the money. Likewise, if we have to evaluate the value of money that we are expected to receive in the future in today’s terms, then we have to move the future money back to today’s terms. This is called the \textbf{Present Value (PV)} of money.

\dfn{Compounding and Discounting}{This process of adjusting the money we have today to calculate its future value is called \textbf{Compounding} and when we have to calculate its present value of some money we are to receive in the future is called \textbf{Discounting}}

Future value can be calculated using:
\begin{displaymath}
  \textrm{Future Value} = \textrm{Amount} \times (1 + \textrm{opportunity cost rate})^\textrm{Number of Years}
\end{displaymath}

Present value can be calculated using:
\begin{displaymath}
  \textrm{Present Value} = \frac{\textrm{Amount}}{(1 + \textrm{Discount Rate})^\textrm{Number of Years}}
\end{displaymath}

\dfn{Net Present Value}{The sum of all present values of the future cash flow is called the \textbf{Net Present Value (NPV)}.}

\subsection{The Free Cash Flow (FCF)}
\dfn{Free Cash Flow (FCF)}{The free cash flow is the excess operating cash that the company generates after accounting for capital expenditures such as buying land, building and equipment.}

This is the cash that shareholders enjoy after accounting for the capital expenditures. The mark of a healthy business eventually depends on how much free cash it can generate.

Thus, the free cash is the amount of cash the company is left with after it has paid all its expenses including investments.

\begin{displaymath}
  \textrm{FCF} = \textrm{Cash from Operating Activities} - \textrm{Capital Expenditures}
\end{displaymath}

\subsection{Key Steps of DCF Analysis}
\begin{enumerate}
  \item Estimate the average free cash flow
  \item Identify the growth rate
  \item Estimate the future cash flows
\end{enumerate}

\subsection{The Terminal Value}

\dfn{Terminal Growth Rate}{The rate at which the company generates free cash flow grows beyond 10 years is called the terminal growth rate.}
\nt{Usually, the terminal growth rate is considered to be less than 5\%}

\dfn{Terminal Value}{The terminal value is the sum of all the future cash flow, beyond the 10th year, also called the terminal year.}

The terminal value can be calculated by taking the cash flow of the 10th year and grow it at the terminal growth rate.
\begin{displaymath}
  \textrm{Terminal Value} = \textrm{FCF} \times \frac{(1 + \textrm{Terminal Growth Rate})}{(\textrm{Discount Rate} - \textrm{Terminal Growth Rate})}
\end{displaymath}

\nt{The FCF used in the terminal value calculation is that of the 10th year.}

\subsection{The Share Price}
The share price we will be talking about here is not the actual share price on the market but the 'intrinsic value' of the share that we wish to find (the valuation of the company).

\subsubsection{Net Debt}
\dfn{Net Debt}{
  \begin{displaymath}
    \textrm{Net Debt} = \textrm{Current Year Total Debt} - \textrm{Cash \& Cash Balance}
  \end{displaymath}
}

\nt{A negative sign indicates that the company has more cash than debt.}

This value must be subtracted from the free cash flow to yield the \textbf{total present value of the free cash flow}.

\begin{displaymath}
  \textrm{Share Price} = \frac{\textrm{Total Present Value of Free Cash Flow}}{\textrm{Total Number of Shares}}
\end{displaymath}

\subsection{Modelling Error \& The Intrinsic Value Band}
Though quite scientific, the DCF model makes a bunch of assumptions and hence would most likely lead to errors. Hence, we should accommodate for modelling errors.

One may allow $\pm 10\%$ leeway in the price.
