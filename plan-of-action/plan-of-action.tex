\documentclass[12pt, letterpaper]{article}

\usepackage[colorlinks=true, urlcolor=blue, linkcolor=red]{hyperref}

\title{Plan Of Action - SoS '24}
\author{Geet Sethi}
\date{May 2024 - July 2024}

\begin{document}

\maketitle

\section{End Goals}

\subsection{Stock Market Fundamentals}
\begin{itemize}
    \item Gain a comprehensive understanding of how stock markets operate, including key concepts like stocks, bonds, commodities, market indices, and trading exchanges.
    \item Learn the different types of parameters and what we can deduce about the stock/organization from them (which will later be helpful in devising better strategies).
    \item Gain a better understanding of how risk management works.
\end{itemize}

\subsection{Mathematics behind Finance}
\begin{itemize}
    \item Learn how the prices of different types of derivatives are calculated and a brief history of how those algorithms evolved.
    \item Learn how the different types of parameters are calculated, which are used to judge the stocks.
    \item Gain a comprehensive understanding of the math that is used to reduce risks (such as in hedge funds)
\end{itemize}

\subsection{Algorithmic Strategies}
\begin{itemize}
    \item Gain knowledge about how different trading algorithms have come up and learn how to create one of my own.
    \item Learn how different trading strategies are evaluated and compared.
    \item Learn how back-testing works (also create a mini back-tester if time permits)
\end{itemize}

\section{Tentative Timeline}

\begin{itemize}
    \item \textbf{Week 1:} Learn about stock market fundamentals.
    \item \textbf{Week 2 \& 3:} Learn about the different mathematical concepts and try to implement some important ones in Python.
    \item \textbf{Week 4:} Learn about managing risks in a portfolio and the different mathematical models devised for it.
    \item \textbf{Week 5 \& 6:} Learn about different trading strategies and core ideas behind those algorithms.
    \item \textbf{Week 7:} Learn about different metrics used to evaluate strategies and the core model of back-testers.
    \item \textbf{Week 8:} Formulate and implement a trading strategy in Python, evaluate it, and try to figure out ways to improve it.
\end{itemize}

\section{Resources}

\subsection{Videos}
\begin{itemize}
    \item Stock market fundamentals - \href{https://youtu.be/PQqfeyUQbyE}{Zerodha Varsity}
    \item Fundamental Analysis - \href{https://youtu.be/PQqfeyUQbyE}{YT Playlist by CNBC}
    \item Technical Analysis
\end{itemize}

\subsection{Books}
\begin{itemize}
    \item \href{https://github.com/zslucky/algorithmic_trading_book/blob/master/sat-ebook-20150618.pdf}{Successful Algorithmic Trading by Michael L. Halls-Moore}
    \item \href{https://github.com/zslucky/algorithmic_trading_book/blob/master/aat-ebook-20170711.pdf}{Advanced Algorithmic Trading by Michael L. Halls-Moore}
\end{itemize}

\subsection{Code}
\begin{itemize}
    \item Implementation of famous trading strategies - \href{https://github.com/je-suis-tm/quant-trading}{GitHub}
\end{itemize}

\end{document}
